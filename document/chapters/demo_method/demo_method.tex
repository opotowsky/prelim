\chapter{Methodology and Demonstration}
\label{ch:demo_method}

This chapter first covers the methodology of the proposed work by introducing
each experimental component and following up with a demonstration of each
component. This has been split into three sections, summarized below.

Section \ref{sec:training} discusses how the training data set is obtained.
This is a set of observations with known inputs, i.e., labels that are to be
predicted. After the initial training data is simulated in Section
\ref{sec:snfsim}, with a possible information reduction step in Section
\ref{sec:inforeduc}, the data set will be input to a statistical learner for
the next step: training a model.

Section \ref{sec:statmodel} is about which algorithms use the training set.
They use the features and labels in the training data to formulate a model.
Algorithm choice and parameters are discussed in Section \ref{sec:choice}.
Next, the main goal for these \gls{ML} models is to supply reactor
parameters associated with some unknown \gls{SNF}. Section \ref{sec:rxtrparam}
shows the results of testing this goal: the prediction of a new instance that
has only features and no burnup label.  

Finally, the algorithms are evaluated for accuracy and validated, as shown in
Section \ref{sec:valid}. In practice, validation implies more than just
ensuring the models are properly fit to the data.  Perhaps the training set was
not representative of the actual data space, whereas non-statistical methods do
not rely on the data space for results. To both understand the performance of
the models, the results are then evaluated for over- or under-fitting. 

\begin{figure}[!htb]
  \centering
  \includegraphics[width=0.8\linewidth]{./chapters/demo_method/methodology.png}
  \caption{Methodology of Proposed Experiment}
  \label{fig:method}
\end{figure}

The demonstrations of the above are based on previous work on the subject
\cite{dayman_feasibility_2013} regarding \gls{ML} model performance
with respect to information reduction.  Replicating the methodology helps to
establish some baseline expectations of reactor parameter prediction and how
the different algorithms perform. 

Next, this work will expand upon the previous work.  The first is adding a
different information reduction technique via gamma energies from the \gls{SNF}
nuclide recipes.  Following this, one could apply a \gls{DRF} that calculates
various spectra based on the types of gamma detectors available to the
forensics community (The information reduction step is not demonstrated here).
Secondly, a more advanced \gls{ML} algorithm, support vector regression, is
included to compare a complex model against the two simpler models.  A
schematic of the workflow involving the experimental components is shown in
Figure \ref{fig:method}.

\section{Training Data}
\label{sec:training}
\subsection{Spent Nuclear Fuel Simulations}
\label{sec:snfsim}

Because creating databases from real measurements to represent reactor
technologies from around the world is impossible, the database in this study
will be created from high-fidelity simulations via \gls{ORIGEN} \cite{origen},
an activation and depletion code within the SCALE 6.2 modeling and simulation
suite \cite{scale}. Specifically, the ARP module of the activation and
depletion code ORIGEN was used: \gls{ORIGEN-ARP}.

A set of simulations of \gls{SNF} at different burnups and cooling times will
comprise the database.  Of interest to an entity trying to create a weapon is
partially irradiated fuel if they have plutonium separations capabilities or
any radioactive substance in the case of a dirty bomb.  Addressing the former,
a smaller burnup than is typical for spent fuel from a commercial reactor is
used in the previous work.  

\begin{table}[!hp]
  \centering
  \begin{subtable}{\linewidth}
    \centering
    \includegraphics[height=0.7\textheight]{./chapters/demo_method/TrainData.png}
    \caption{Reactor types and uranium-235 enrichment [weight\%]}
    \label{tbl:rxtrtype}
    \vspace*{5mm}
  \end{subtable}
  \begin{subtable}{\linewidth}
    \centering
    \includegraphics[width=0.7\linewidth]{./chapters/demo_method/TrainData2.png}
    \caption{Simulation space defining reactor parameters and cooling time}
    \label{tbl:rxtrparam}
  \end{subtable}%
  \caption{Design of the training set space}
  \label{tbl:train}
\end{table}

It should be noted that many algorithms are developed on an assumption that the
training set will be \gls{i.i.d.}. This is important so that the model does not
overvalue or overfit a certain area in the training space.  A truly
\gls{i.i.d.} training set would go beyond the lower burnups, but this is purely
for demonstration with a single use case in mind.  The training database is
thus constructed by simulating the same training set space as described in Ref.
\cite{dayman_feasibility_2013}, shown in Table \ref{tbl:train}. For each entry 
shown here the simulations included\todo{oops, finish!} 

\begin{table}[!hp]
  \centering
  \includegraphics[width=0.95\linewidth]{./chapters/demo_method/TestData.png}
  \caption{Design of the testing set space}
  \label{tbl:test}
\end{table}

While in most machine learning studies the testing set is chosen randomly from
the training set, the previous work used an external one, shown in Table
\ref{tbl:test}.  Although the test set was designed to have values in between
the trained values of burnup, it was chosen systematically. Therefore it was
implemented in this study for comparison, but cross-validation will be used
moving forward. More specifically, using \textit{k}-fold cross-validation is
expected to better indicate the model performance. 

\subsection{Information Reduction}
\label{sec:inforeduc}

Since the overall goal of this project is to determine how much information to
what quality is needed to train a machine-learned model, there will be an 
information reduction manipulation applied to the training data set. This study 
evaluates the impact of randomly introduced error of varying amounts on the 
ability of the algorithms to correctly predict the burnup. 

The three algorithms will be evaluated with error applied to each nuclide
vectors in the training set.  A maximum error is ranging from $0 - 10\%$ is
chosen for each round of training, and a random error within the range of
$[1-E_{max}, 1+E_{max}]$ is applied to each component of the nuclide vector.

However, error in a nuclide vector is not random, in fact it is systematic and
dependent on a number of known sources of uncertainty. The next study will
introduce error by limiting the nuclides to only those that can be measured
with a gamma spectrometer. Although this is initially done using the
availability of gamma spectra in \gls{ORIGEN}, \gls{GADRAS} can provide more
\gls{DRF}s to further reduce information given to the algorithm.




\section{Statistical Learning for Models}
\label{sec:statmodel}
\subsection{Algorithms Chosen}
\label{sec:choice}

Choosing which algorithms to test is largely intuitive. It is based on the
strengths and weaknesses of different optimization methods within the
algorithms as well as what is being predicted.  

For a benchmarking excercise, some \gls{ML} approaches here were chosen
based on previous work \cite{dayman_feasibility_2013}: nearest neighbor and
ridge regression. These are useful because they are simple, providing
distance-based and linear-based models, respectively. If more complex
algorithms are not required to obtain useful results, then there is no need to
use more computationally expensive options. However, hedging on the fact that
more complex models will be needed, this work also employs an algorithm that is
known to handle highly dimensional data sets well: support vector regression.
These algorithms were introduced in Section \ref{sec:algs}. 

\begin{table}[!htb]
  \centering
  \includegraphics[width=0.8\linewidth]{./chapters/demo_method/defaults.png}
  \caption{Algorithm Parameters Used in Demonstration}
  \label{tbl:defaults}
\end{table}

The parameters chosen for the algorithms are shown in Table \ref{tbl:defaults}.
For many, the default behavior was retained. Prior to training, the data set is
preprocessed by scaling and normalization because the nuclide concentrations
vary by many orders of magnitude. Algorithm implementations from a python-based
\gls{ML} toolkit, scikit-learn \cite{scikit}, are used to train the
models.

\subsection{Reactor Parameter Prediction}
\label{sec:rxtrparam}

The above was carried out, training various statistical models of \gls{SNF}
using nuclide correlations with burnup. As a reminder, this is not the entirety
of the nuclode output from \gls{ORIGEN}; it is the top 200 nuclides by
concentration in each row. Following the training phase, it is important to
estimate the reactor parameter prediction capabilities of those models by
testing their generalizability.  This is done with the previously described set
of measurements from a test data set (shown in Table \ref{tbl:test}) with
samples that mimic potential interdicted \gls{SNF}. The testing set has the
same features as the training set, with known burnup labels that are compared
to the predicted labels. 

\begin{figure}[!htb]
  \centering
  \includegraphics[width=\linewidth]{./chapters/demo_method/randerr.png}
  \caption{Prediction Error from Information Reduction via Random Error}
  \label{fig:randerr}
\end{figure}

Next, information reduction was carried out with all three \gls{ML} approaches.
Figure \ref{fig:randerr} shows the three algorithms' \glspl{MAPE} with respect
to the reduction of information by the introduction of random error to the
nuclide vectors, as described in Section \ref{sec:inforeduc}.  \gls{SVR} is
shown to perform the best, but it quickly reaches 100\% error. \todo{Discuss
mape or change error here?}  Although ridge regression rapidly increases with
any amount of error, nearest neighbor regression shows more promise, although
it reaches 257\% at 10\% error.  Overall, this performance indicates that these
algorithms are unlikely to predict burnup when faced with the uncertainties in
real-world measurements or the further reduction in information from gamma
detection.

\begin{table}[!htb]
  \centering
  \includegraphics[width=0.8\linewidth]{./chapters/demo_method/results1.png}
  \caption{Three Models' Burnup Prediction Errors}
  \label{tbl:err}
\end{table}

For no introduced error, each model's prediction errors are shown in Table
\ref{tbl:err}.  First, shown here are two \textit{sources} of error: from the
testing set and via cross-validation.  Although previous work uses the former,
it is expected that the latter will provide better estimates of model behavior.
The models evaluated by the testing set do not have a validation set for
pre-evaluation.  The models that are evaluated via cross-validation do not use
the testing set. 

Table \ref{tbl:err} also includes two error \textit{metrics}.  For the sake of
comparison to previous work and convenient interpretation, \gls{MAPE} is
tracked. However, \gls{MAPE} is not the only error metric. Being that it
approaches infinity near true values of zero, absolute and squared deviations
can also be tracked and may provide more information.  Anecdotally, the
preferred method in the community is to use \gls{RMSE} for model error
estimation, so both are tabulated.  Regardless, the \gls{MAPE} shows that there
are some extremely high and extremely low errors depending on the algorithm;
both results indicate poor performance.  This is quite concerning, because
these results are from error-free data.

Despite this poor initial performance, the results in Table \ref{tbl:err} and
Figure \ref{fig:randerr} can be evaluated. Next introduced are some diagnostic
and optimization procedures that can shed light on these results.


\section{Validation}
\label{sec:valid}
\subsection{Model Diagnostics}
\label{sec:algeval}

Machine learning algorithms are heavily dependent on the inputs and parameters
given to them, such as training set sizes, regularization, learning rates, etc.
From the results shown in Section \ref{sec:statmodel}, it is clear there is
room for improvement.  The diagnostic plots show the errors of the predicted
burnup values to the actual burnup values with respect to some variable on the
\textit{x}-axis.  As previously introduced in Section \ref{sec:optvalid}, the
errors are compared to the training error to understand the generalization
strength with respect to training set size (learning curves) and the algorithm
parameters governing model complexity (validation curves). 

In addition to machine learning best practices, another layer of comparison is
added here.  Because it is difficult to ensure consistently representative
testing data, the accuracy of a learned model should not depend on only one
testing set.  The learned model's accuracy is better estimated by using a
validation set, or even better, \textit{k}-fold cross-validation, introduced in
Section \ref{sec:selectass} This work includes both the testing error (using
the testing set described in Section \ref{sec:training}) and cross-validation
error. The predetermined testing set will allow for comparison against the
previous work it was obtained from, but it is assumed that cross-validation
will provide a better indication of model performance.

\begin{figure}[!htb]
    \centering
    \includegraphics[width=\linewidth]{./chapters/demo_method/lc1.png}
    \caption{Learning curve for burnup prediction, $\gamma = 0.001$}
    \label{fig:lc1}
\end{figure}

The learning curves are obtained as follows, shown in Figure \ref{fig:lc1}.
For a given (randomly chosen) training set size between 15 and 100\% of the
total data set, training and prediction rounds were performed for each. The
testing error scenario performs this \textit{k} times and averages those
results.  This is equivalent to the \textit{k} in \textit{k}-fold
cross-validation to provide some semblance of equivalent statistics.  The
cross-validation error scenario has no need for averaging because it is
performed automatically. In both cases, the learning curves do not provide a
clear picture of over- or undertraining upon first glance.

\begin{figure}[!htb]
    \centering
    \includegraphics[width=\linewidth]{./chapters/demo_method/vc1.png}
    \caption{Validation curve for burnup prediction, $TrainSize = 2313$}
    \label{fig:vc1}
\end{figure}

The validation curves are obtained as follows, shown in Figure \ref{fig:vc1}.
The $\gamma$ parameter in \gls{SVR}, which influences model complexity, was
varied from $10^{-4}$ to $10^{-1}$. Training and prediction rounds were
performed for different $\gamma$ values in this range.  Again, the testing and
cross-validation errors are both used as described above. As with Figure
\ref{fig:lc1}, determining the robustness to over- or undertraining is
difficult here although there is possibly a minimum at $\gamma = 0.001$.

Although there is no example behavior of Figure \ref{fig:lc1}'s peculiar
learning curve in Figure \ref{fig:learning}, the curve mimics the squared bias
curve from Figure \ref{fig:bvtradeoff}. This indicates that the bias in the
model is much higher than the variance.  Next, the testing error is lower than
the training error; this should never be the case, and indicates an issue with
the systematically chosen testing set.  While the cross-validation error is
correctly higher than the training error, it follows along in parallel,
producing no information on model fitness other than confirming a very high
bias.  It is presumed this is not the fault of the algorithms, but the training
set itself.  It is likely covering too small of a range of the simulation
space.

Additionally, Figure \ref{fig:vc1}'s validation curve shows the testing error
dropping below the training error for extremely small $\gamma$, around where a
minimum might be.  Since the model suffers from high bias, no amount of model
complexity can be optimized. 

%%%%%%%%%%%%%%%%%%%%%%%%%%%%%%%%%%%%%%%%%%%%%%%%%%%%%%%%%%%%%%%%%%%%%%%%%%%%%
%%%%%%%%%%%%%%%%%%%%%%%%%%%%%%%%%%%%%%%%%%%%%%%%%%%%%%%%%%%%%%%%%%%%%%%%%%%%%
The resolution to the underfitting is discussed
To obtain reliable models, one must both choose or create a training set
carefully and study the impact of various algorithm parameters on the error.
Many algorithms are developed on an assumption that the training set will be
independent and identially distributed (i.i.d.). This is important so that the
model does not overvalue or overfit a certain area in the training space. The
testing error can therefore be tabulated with respect to training set size,
number of features, or algorithm parameters (regularization terms, etc). The
results are broadly known as diagnostic plots. 

The next step is to provide a larger, more diverse training set to the
algorithms so they could predict better when faced with new instances.  Dayman
training/test set -> sfcompo-like sims and testing set
%%%%%%%%%%%%%%%%%%%%%%%%%%%%%%%%%%%%%%%%%%%%%%%%%%%%%%%%%%%%%%%%%%%%%%%%%%%%%
%%%%%%%%%%%%%%%%%%%%%%%%%%%%%%%%%%%%%%%%%%%%%%%%%%%%%%%%%%%%%%%%%%%%%%%%%%%%%

Lastly, it must be noted that the errors here are all measured in \gls{MAPE}.
While this is more intuitively convenient, it requires ensuring that no true
values are $0$. Thus, \gls{RMSE} is the preferred error measurement used in
the community and will be used in all future work.

%%%%%%%%%%%%%%%%%%%%%%%%%%%%%%%%%%%%%%%%%%%%%%%%%%%%%%%%%%%%%%%%%%%%%%%%%%%%%
%%%%%%%%%%%%%%%%%%%%%%%%%%%%%%%%%%%%%%%%%%%%%%%%%%%%%%%%%%%%%%%%%%%%%%%%%%%%%
\subsection{Model Comparison}
\label{sec:algcompare}
%%%%%%%%%%%%%%%%%%%%%%%%%%%%%%%%%%%%%%%%%%%%%%%%%%%%%%%%%%%%%%%%%%%%%%%%%%%%%
%%%%%%%%%%%%%%%%%%%%%%%%%%%%%%%%%%%%%%%%%%%%%%%%%%%%%%%%%%%%%%%%%%%%%%%%%%%%%

Regression Training Error: Confidence intervals on predictions to understand
true error versus sample error Test set must be > 30 instances, Can easily
calculate N\% confidence interval.

Options for comparison of algorithms: inverse bayesian stuff, comparing
classification of 2 classes on same ROC plot with multiple ML systems, Scatter
plots, Pairwise t-tests.
\todo{Discuss understanding confidence intervals in predictions.}

Alg Compare: inverse prob theory
In addition to evaluating a single learned model, it may be beneficial to
compare models. As discussed in \todo{the algorithm comparison section}, there
are three methods that will be used: comparison of \gls{ROC} curves, scatter
plots, and pairwise \textit{t}-tests.



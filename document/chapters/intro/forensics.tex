The process of technical nuclear forensics includes the analysis and
interpretation of nuclear material to determine its history, whether that be
intercepted \gls{SNF}, \gls{UOC}, or the debris from an exploded nuclear
device. After the technical portion is complete, intelligence data can be used
to aid in material attribution; this is the overall goal of nuclear forensics. 

After a nuclear incident, the material or debris is sampled and evaluated
through many techniques that provide the following information: material
structure, chemical and elemental measurements, and radioisotope measurements.
These measurements or calculated ratios comprise the forensic signatures of the
sample in question. These signatures can be analyzed with specific domain
knowledge; for example, \gls{UOC} will have trace elements depending on where
it was mined from.  They can also be analyzed against a forensics database, in
the case of \gls{SNF}.

Measurement needs and techniques vary vastly depending on the material, as does
the type of signature. This study focuses on non-detonated materials,
specifically, \gls{SNF}. It is important to determine if some intercepted
material is from an undisclosed reactor or a commercial fuel cycle to attribute
it to an entity or state. This is typically done by obtaining chemical and
elemental signatures as well as isotopic ratios, and comparing these
measurements to those in an existing forensics database of reference \gls{SNF}. 

The signatures for \gls{SNF} correlate to several characteristics of quantities
that can, in a best case scenario, point to the exact reactor from which the
fuel was intercepted. The reactor parameters of interest are the reactor type,
fuel type and enrichment at beginning of irradiation, cooling time, burnup.
\todo{need citation, prob ORNL paper} \todo{more deets here? prob better later}

The current and future work of this study are designed based on two primary
needs to bolster the \gls{US} nuclear forensics capability: forensics databases
are imperfect, and our best measurement techniques are not always feasible in
an emergency scenario. It is proposed that using a machine-learned model may be
able to combat these issues, discussed in Section \todo{ref}. 

Forensics databases are kept by individual countries, and a given database will
have widely varying uncertainty depending where and on what instrument the
material was measured. Additionally, some fields have missing data. This
presents issues with matching and characterizing \gls{SNF} based on
interpolating between entries. \todo{more deets and cite}

A lofty goal for the forensics community would be to develop methods that
provide instantaneous information, reliable enough to guide an investigation
(e.g., within 24 hours). In the case of \gls{SNF}, it takes weeks in a lab to
measure isotopes via advanced gamma spectroscopy and mass spectrometry
equipment. Thus, fast measurements to provide isotopic ratios to calculate the
above-mentioned fuel parameters of interest would provide this via some form of
a handheld detector that measures gamma spectra.  However, while this
nondestructive analysis is rapid, it is also difficult to evaluate because of
the presence of overlapping peaks. Thus, gamma spectra give less information at
a higher uncertainty than the near-perfect results of some destructive mass
spectrometry techniques, like TIMS\todo{find MS technique and citation}.
Additionally, within gamma spectroscopy techniques (e.g., field vs. lab
detectors), uncertainties can vary significantly because of the detector
reponse, environment, storage, electronics, etc. \todo{fix up and summary}


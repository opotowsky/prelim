\chapter{Introduction}
\label{ch:intro}

Fill me in, yo.

\section{Motivation}
\label{sec:motivation}

In the event of a nuclear incident, such as the retrieval of stolen nuclear
material or the detonation of a dirty bomb, it is necessary to learn as much as
possible about the source of the materials in a timely manner. In the case of
non-detonated special nuclear material, knowing the reactor parameters that
produced it can point investigators in the right direction in order to
determine the chain of custody of the interdicted material. Determining these
parameters (e.g., reactor type, cooling time, burnup) requires first
characterizing and calculating certain isotopic ratios, chemical compounds, or
trace elements.  Both radiological methods (e.g., gamma spectroscopy) and
ionization methods (e.g., mass spectroscopy) measure these quantities. Although
both measurement techniques have a multitude of techniques within them and thus
varying strengths and weaknesses, the main tradeoff is between time/cost and
amount of information gained. 

The results of these analytic techniques are then compared against existing
databases to obtain the desired reactor parameters. These databases are highly
multidimensional, and furthermore, are rife with missing data entries and
inconsistent uncertainties. Direct comparison between measurement results and a
database therefore may not yield accurate results. Thus, computational
techniques have been developed by nuclear engineers to calculate the parameters
relevant to nuclear forensics analysis. \todo{Cite inverse stuff, possibly
adjoint?} Another approach with the uniqueness of requiring minimal domain
knowledge is the use of statistical methods via machine learning algorithms to
predict these characteristics or values. These algorithms can create a model
using the database entries that enables "filling between the lines" of its
entries. Additionally, having a machine-learned model may overcome the above
challenges of multidimensionality, missing data, and irregular uncertainty.

\subsection{Needs in Nuclear Forensics}
The process of technical nuclear forensics includes the analysis and
interpretation of nuclear material to determine its history, whether that be
intercepted \gls{SNF}, \gls{UOC}, or the debris from an exploded nuclear
device. After the technical portion is complete, intelligence data can be used
to aid in material attribution; this is the overall goal of nuclear forensics. 

After a nuclear incident, the material or debris is sampled and evaluated
through many techniques that provide the following information: material
structure, chemical and elemental measurements, and radioisotope measurements.
These measurements or calculated ratios comprise the forensic signatures of the
sample in question. These signatures can be analyzed with specific domain
knowledge; for example, \gls{UOC} will have trace elements depending on where
it was mined from.  They can also be analyzed against a forensics database, in
the case of \gls{SNF}.

Measurement needs and techniques vary vastly depending on the material, as does
the type of signature. This study focuses on non-detonated materials,
specifically, \gls{SNF}. It is important to determine if some intercepted
material is from an undisclosed reactor or a commercial fuel cycle to attribute
it to an entity or state. This is typically done by obtaining chemical and
elemental signatures as well as isotopic ratios, and comparing these
measurements to those in an existing forensics database of reference \gls{SNF}. 

The signatures for \gls{SNF} correlate to several characteristics of quantities
that can, in a best case scenario, point to the exact reactor from which the
fuel was intercepted. The reactor parameters of interest are the reactor type,
fuel type and enrichment at beginning of irradiation, cooling time, burnup.
\todo{need citation, prob ORNL paper} \todo{more deets here? prob better later}

The current and future work of this study are designed based on two primary
needs to bolster the \gls{US} nuclear forensics capability: forensics databases
are imperfect, and our best measurement techniques are not always feasible in
an emergency scenario. It is proposed that using a machine-learned model may be
able to combat these issues, discussed in Section \todo{ref}. 

Forensics databases are kept by individual countries, and a given database will
have widely varying uncertainty depending where and on what instrument the
material was measured. Additionally, some fields have missing data. This
presents issues with matching and characterizing \gls{SNF} based on
interpolating between entries. \todo{more deets and cite}

A lofty goal for the forensics community would be to develop methods that
provide instantaneous information, reliable enough to guide an investigation
(e.g., within 24 hours). In the case of \gls{SNF}, it takes weeks in a lab to
measure isotopes via advanced gamma and mass spectroscopy equipment. Thus, fast
measurements to provide isotopic ratios to calculate the above-mentioned fuel
parameters of interest would provide this via some form of a handheld detector
that measures gamma spectra.  However, while this nondestructive analysis is
rapid, it is also difficult to evaluate because of the presence of overlapping
peaks. Thus, gamma spectra give less information at a higher uncertainty than
the near-perfect results of some destructive mass spectroscopy techniques, like
TIMS\todo{find MS technique and citation}.  Additionally, within gamma
spectroscopy techniques (e.g., field vs. lab detectors), uncertainties can vary
significantly because of the detector reponse, environment, storage,
electronics, etc. \todo{fix up and summary}



\subsection{Contribution of Statistical Methods}
As previously mentioned, there are two main issues that are being addressed for
forensics of \gls{SNF}: database issues and speed of characterization. Many
have begun considering computational approaches to nuclear forensics problems,
such as the INDEPTH tool for inverse depletion and decay analysis
\cite{weber_2006, weber_2010, weber_2011}. This tool uses an iterative
optimization method involving many forward simulations to obtain reactor
parameters of interest given some initial values. 

Another approach utilizes artificial intelligence to solve nuclear forensics
problems, such as implementing searching algorithms for the database comparison
step \cite{gey_search} and machine learning for determining reactor parameters
from \gls{SNF} characteristics \cite{dayman_feasibility_2013, nicolaou_2006,
nicolaou_2009, nicolaou_2014, robel_2009, jones_viz_2014, jones_snf_2014}.  A
variety of statistical and machine learning tools have been used to
characterize spent fuel by predicting categories or labels (reactor type, fuel
type) as well as predicting values (burnup, initial enrichment, or cooling
time) The former uses classification algorithms and the latter uses regression
algorithms. Many algorithms can be applied to both cases.

A typical (supervised) machine learning workflow would take a set of training
data with labels or values inserted into some statistical learner, calculate
some objective, minimize or maximize that objective, and provide some model
based on that output. Then a test set (with known values) is provided to the
model so that its performance can be evaluated and finalized. After model
finalization, a user can provide a single instance and a value can be predicted
from that. \todo{insert ML schematic}

To obtain reliable models, one must 1. choose/create a training set carefully
and 2. study the impact of various algorithm parameters on the error. Many
algorithms are developed on an assumption that the training set will be
independent and identially distributed (i.i.d.). [Aside: there are ways to
handle skewed data sets] This is important so that the model does not overvalue
or overfit a certain area in the training space. Additionally, algorithm
performance (or error) can be optimized with respect to training set size,
number of features, or algorithm parameters (regularization terms, etc).  These
are known as diagnostic plots. When plotting the training and testing error
with respect to the number of instances, this is known as a learning curve.
When plotting these errors with respect to the number or features or algorithm
parameters, this is known as a validation curve. \todo{insert example
diagnostic plot?}

Algorithm choice is usually based on what is being predicted and intuition
regarding strengths and weaknesses.  For the sake of comparison (i.e. weak
validation), some machine learning approaches here are based on previous work
\cite{dayman_feasibility_2013} while also extending to a more complex model
via an algorithm that is known to handle highly dimensional data sets well.
Thus, this paper investigates three regression algorithms: nearest neighbor,
ridge, and support vectors.


It is first important to determine if statistical methods can
overcome the inherent database deficiencies. Next, the statistical methods must
be considered in such a way as to represent a real-world scenario. Although
mass spectrometry techniques provide extremely accurate isotopic information
for analytical methods, they are time-consuming and more expensive. And
although gamma spectroscopy can give extremely fast results cheaply, it only
measures certain radiological signals and is influenced by many environmental
factors, storage, and self-attenuation. As different machine learning
algorithms and parameters are investigated, this work focuses on probing the
amount of information required to obtain realistic results.

Because creating databases from real measurements to represent reactor
technologies from around the world is impossible, the database in this study
will be created from high-fidelity simulations via ORIGEN irradiation and
depletion \todo{check actual name of code part used}. In the simulation and
statistical learning paradigm, we need to determine how much information to
what quality is needed to train a machine-learned model; the model must give
appropriate predictions of reactor parameters given a set of measurements from
a test sample of interdicted \gls{SNF}. Of interest to an entity trying to
create a weapon is partially irradiated fuel if they have plutonium separations
capabilities or any radioactive substance in the case of a dirty bomb.
Addressing the former, a set of simulations of \gls{SNF} at different burnups
and cooling times will comprise the database.\todo{rewrite to be clearer}

Can the algorithm overcome the deficiencies of gamma detection and still
provide useful results? Or does it need more information, e.g., exact
isotopics? First, we must establish some baseline expectations of reactor
parameter prediction and how different algorithms perform. This work is based
off previous work on the subject \cite{dayman_feasibility_2013} regarding
machine learning performace with respect to information reduction, and expands
upon it by also evaluating a more advanced machine learning algorithm: support
vector regression. 



Below is a more in depth discussion of nuclear forensics and
how machine learning can contribute to this research area. After that, an
experimental design is outlined. Lastly, the results are presented and
discussed. 

Thus, ultimately, the goal is to answer the question \textit{How
does the ability to determine forensic-relevant spent nuclear fuel attributes
degrade as less information is available?}. 

%%%%%




\section{Methodology}
\label{sec:methodology}

Talk about workflow here. Might need separate tex file. Prob should keep pretty
simple since everything is discussed later in the demonstration part of the
prelim. 

\section{Goals}
The main purpose of this work is to evaluate the utility of statistical methods
as an approach to determine nuclear forensics-relevant quantities as less
information is available. Machine learning algorithms will be used to train
models to provide these values (e.g., reactor type, time since irradiation,
burnup) from the available information. The training data will be simulated
using the ORIGEN tool \todo{cite}, which will provide an array of nuclide
concentrations as the features ($X$) and the value of interest ($y$) is
provided from the simulation inputs.  Information reduction will be carried out
using computationally generated gamma spectra; the radionuclide concentrations
from the simulations can be converted into gamma energies, which then undergo a
detector response calculation to represent real gamma spectra as closely as
possible. Machine learning best practices will be used to evaluate the
performance of the chosen algorithms, and established statistical methods will
be used to provide an interval of confidence in the model predictions
\todo{correct lang?}.

The necessary background is covered in Chapter \ref{ch:litrev}.  First, an
introduction to the broader field of nuclear forensics is in Section
\ref{sec:nfoverview} to place this work in the context of the technical mission
areas. After that, a short discussion of the field of machine learning, the
algorithms used, and validation methods are in Section \ref{sec:mlback}. A
review of statistical methods being used in studies of forensics analysis is
covered next in Section \ref{sec:stats4nf}. Lastly, Section \ref{sec:fcsim}
includes information about the codes used to generate the training data, via
fuel cycle simulation, detector response function, and isotope identification
of gamma spectra. 

After the existing work is discussed and the gap that this work will fill is
identified, the experimental components are introduced next in Chapter
\ref{ch:method}. A demonstration of the methodology is presented next in Chapter
\ref{ch:demo}. Following these two chapters, Chapter xx %\ref{ch:proposal} 
describes the thesis research proposal and corresponding hypotheses. 
Finally, future directions and alternative directions are identified in 
Chapter xx. %\ref{ch:future}


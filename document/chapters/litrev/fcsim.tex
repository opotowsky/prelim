There can be a large number of computational tools within a given field of
study.  They are typically developed with a single aspect of analysis in mind,
so one must understand the simplifying assumptions the developers made for
their purposes. Thus, choosing a computational tool is not always obvious. This
section introduces the choices for the various experimental components.

\subsection{Fuel Cycle Simulation}

Nuclear fuel cycle studies involve tracking the material flow of nuclear fuel.
This can be anywhere from mining to waste management, or focus on a process
step in between. Fuel cycle studies are not necessarily nuclear-specific. For
example, they can be used to evaluate economic predictions, environmental
impact, transportation planning, etc.  In order to draw conclusions from these
studies, it is common to use a nuclear fuel cycle simulator that tracks the
quantities of interest. These allow the comparison of different fuel types,
reactor technologies, material processing steps, etc. 

There are simplifications researchers need to make in order to experiment in a
controlled way. Fuel cycle simulators, built for a specific needs, must remove
complicating factors that are less relevant to the study.  For example, one
tool might be suited well to large-scale systems analysis with little nuclear
physics included in the models, and another might focus on detailed isotopics
within a system to track plutonium.

Because a large portion of a nuclear forensics investigation relies on
measuring isotopics, this work used \gls{ORIGEN} \cite{origen}, which is a part
of the \gls{SCALE} 6.2 modeling and simulation suite of computational tools
developed for nuclear design and safety \cite{scale}. \gls{ORIGEN} was chosen
for its physically detailed models of activation, depletion, and decay.
Specifically, the ARP module of the code was used: \gls{ORIGEN-ARP}.

\gls{ORIGEN} calculates time-dependent nuclide concentrations (or quantities
derived from these) that result from activation and depletion calculations. The
physics (i.e., neutron transport and decay) calculations are carried out in
other \gls{SCALE} modules that solve the depletion equations.  This generates
libraries for \gls{ORIGEN} that include the probabilities of reaction (i.e.,
cross sections) for the system.

To obtain an \gls{SNF} recipe from a reactor simulation, \gls{ORIGEN} uses the
desired input power generation with the cross section library to calculate a
flux, the resulting depletion, and the end composition (i.e., isotopic recipes
or nuclide vectors).  Another output is decay; the composition is computed
using decay equations with nuclear data \cite{endf}. These compositions provide
source terms for other calculations, such as decay emission spectra from
neutrons, alpha particles, beta particles, and gamma rays. Other derived
quantities like activity, decay heat, or radiological hazard factors are also
an option.

\gls{ORIGEN-ARP} allows users to access a wider range of simulations by
interpolating between the pre-calculated libraries instead of creating new
libraries.  It is known to be validated for \gls{LWR} \gls{SNF}
\cite{lwr_valid}. Additionally, recycled \gls{SNF} in the form of mixed oxide
fuel has been benchmarked for the relevant reactors \cite{mox_valid}.  Through
\gls{ORIGEN}, given an initial material composition, some reactor operation
parameters, and a reactor type, one can quickly perform many different nuclear
reactor simulations and obtain \gls{SNF} recipes.

\subsection{Statistics Toolkit}

The statistics toolkit chosen for this work is scikit-learn \cite{scikit}, a
machine learning package in python.  Virtually all modern \gls{ML}
toolkits will have acceptably fast and reliable algorithms, but the use of
python provides a platform for seamless integration of all the tools in the
workflow. 

\subsection{Computational Gamma Spectra}

Although the data modification shown later in Section \ref{sec:inforeduc} does
not use any standalone tool, the code \gls{GADRAS} \cite{gadras} developed at
Sandia National Laboratories will provide information reduction in a physically
valid manner.  Test studies will be carried out using the gamma energies
available from \gls{ORIGEN} simulations, but the detector response will need to
be varied more than this tool can provide. Thus, given the gamma energies from
\gls{ORIGEN} and a user-chosen \gls{DRF}, \gls{GADRAS} computationally
generates gamma spectra.  This will enable a more robust study of statistical
performance with respect to information reduction.


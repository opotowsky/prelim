Machine learning is a sub-field of \gls{AI} within the broad category of
computer science. The goal of \gls{AI} is to create computer systems that
respond to their environment according to some set of criteria or goal. For
example, self-driving vehicles have computers on board that learn to avoid
curbs and humans. It is common knowledge that the use of \gls{AI} has been
expanding at a rapid rate in recent years. News stories of major tech
companies' \gls{AI} advancements are frequent and news articles abound with
data on which jobs will be replaced with \gls{AI} in the near future. 

While its use has been increasing in the commercial sector, there is also much
anecdotal evidence to support the existence of a rapid increase of \gls{AI} use
in academic research across many disciplines beyond robotics. \gls{AI} systems
have been used in detection (e.g., fraud or spam), medical diagnostics, user
analysis (e.g., Netflix ratings), and a host of scientific disciplines that
have increasing amounts of multivariate data.

Much of the recent advances to the field of \gls{AI} have occured in the
statistical realm, which forgoes domain knowledge in favor of large data sets.
Thus, machine learning and statistical learning has become somewhat of a
separate field \cite{changingml}. Machine learning research focuses on the
underlying algorithms using mathematical optimization, methods for pattern
recognition, and computational statistics.  As an application, however, this
study is not concerned with computational time, but rather the ability to
correctly predict values and categories relevant to the nuclear forensics
mission. This restricts the relevancy of the algorithms to the underlying
theory and its impact on the resulting model's accuracy. 

Machine learning algorithms can be separated into two main categories:
unsupervised and supervised learning.  The former groups or interprets a set of
input data, predicting patterns or structures. The latter includes both the
input and output data, enabling the trained model to predict future outputs.
Broadly speaking, the unsupervised learning algorithms are designed for
clustering data sets or dimensionality reduction (i.e., determining some subset
or linear combination of features most relevant to the input data) of data
sets.  Supervised learning algorithms predict both discrete and continuous
values via classification and regression, respectively. Some algorithms can
perform both classification and regression, and neural networks can even be
modified to perform either supervised or unsupervised learning. 
%Additionally, various algorithms can be strung together, which is referred to
%as \textit{ensemble methods}. One common way of doing this is performing
%deimensionality reduction prior to supervised learning
\\
\begin{figure}[!htb]
  \includegraphics[width=\linewidth]{./chapters/intro/SupervisedRegression.png}
  \caption{Schematic representing a statistical learning regression algorithm}
  \label{fig:supervised}
\end{figure}

As shown in Figure \ref{fig:supervised}, a typical (supervised) machine
learning workflow begins with a training data set, which has a number of
\textit{instances}, or rows of observations.  Each instance has some
\textit{attributes}, also referred to as \textit{features}, and a label, which
can be a categorical label or discrete/continuous values.  

The training data are then inserted into a statistical learner; this calculates
some objective, minimizes or maximizes that objective, and provides some model.
This model is typically evaluated using a testing set that has the same set of
attributes and labels (but different instances). The comparison of what the
model predicts and the actual label gives the \textit{generalization error}.
Depending on the performance and application, the model may need improvement
from more training and/or some changes in the algorithm parameters. Once the
model is performing well enough and validated, it is finalized; then a user can
provide a single instance and a value can be predicted from that. 

This study performs regression tasks using supervised learning algorithms.
Differences among the underlying mathematics of the algorithms impact the
trained models.  Therefore the algorithms used in this study will be discussed
in Section \ref{sec:algs}. Next, model selection and assessment is covered in
Section \ref{sec:selectass}.  Evaluating and optimizing algorithm performance
is discussed in Section \ref{sec:optvalid}, as well as robustly comparing
different algorithms for validation.

\subsection{Algorithms for Statistical Learning}
\label{sec:algs}
For relevant nuclear forensics predictions, both classification and regression
algorithms must be used.  For example, one may want to predict the reactor type
based on some measurements (referred to as features) of spent fuel of an
unknown source, and this would require a classification algorithm. Or perhaps
the input fuel composition is relevant to an investigation on weapons intent,
so a regression algorithm would be used to train a model based on some set of
features.  Since algorithm formulation impacts the resulting performance, they
are discussed in detail below.  

\todo{should add useful vocab: training data X and y, instances, features, etc}

\subsubsection{Linear Models}
\label{sec:linear}

One of the simplest and most obvious methods of prediction is a linear model
using a least-squares fit. 

%\subsubsubsection{Ridge Regression}
%\label{sec:ridge}

%not sure about this organization

%\subsubsubsection{Other Linear Stuff?}
%\label{sec:other}

\todo{intro, alg math, explain regularization} 

not sure about this organization

\subsubsection{Nearest Neighbor Methods}
\label{sec:neighbor}

Nearest neighrbor is 

\todo{explain distance metrics} 


\begin{equation}
  \hat{Y}(x) = \frac{1}{k} \sum_{x_i \in N_k(x)} y_i
\end{equation}

Nearest neighbor regression calculates a value based on the instance that is
closest to it. The metrics for distance differ, but in this study, Euclidian
distance was used. There is no learning in this regression, per se; the
training set populates a space and the testing set is compared directly to
that. \cite{elements_stats} 


\subsubsection{Support Vector Machines}
\label{sec:svm}

Support vector regression (SVR) is an extension of the popular classification
algorithm, support vector machine (SVM).  This algorithm was chosen because of
its ability to handle highly dimensional data well, which in this study is
approximately 300 features. 

SVM classifies two classes by determining an optimal hyperplane, given by wx+b,
between them.  As seen in Figure ?, the algorithm evaluates the quality of the
line that separates two classes by maximizing the width of the margin given the
contraints surrounding the line.  Some problems are not linearly separable, and
thus a penalty term is introduced to allow for misclassifications. As shown in
Figure ?, the algorithm then simultaneously minimizes the misclassifications
while maximizing the margin. 

This can be extended easily to multidimensional analysis via what is called the
\textit{kernel trick}.  First, using a nonlinear kernel function maps the data
into higher dimensional feature space. Then the algorithm can find a linear
separation in this space, as shown in Figure ?. Further, this can be upgraded
from classification to SVR by doing similar math but instead minimizing the
margin, as shown in Figure ?. 

The kernel chosen for this study is the Gaussian radial basis function, shown
below. This has two tuneable parameters, gamma and C. Gamma influences the
width of influence of individual training instances, and strongly affects the
fitting of the model. Low values correspond to underfitting because the
instances have too large of a radius (low influence) and high values correspond
to overfitting because the instances have a small radius (high influence). 

The C parameter also affects the fitting of the model by allowing more or less
support vectors, corresponding to more or less misclassification. A lower C
smooths the surface of the model by allowing more misclassifications, whereas a
higher C classifies more training examples by allowing fewer
misclassifications. Too low and too high of a C can cause under- and
overfitting, respoectively. 

Since there is a tradeoff of fitting strength provided by both parameters, it
is common to run the algorithm on a logarithmic grid from 10'-3 to 10'3 for
each parameter. If plotted on a heatmap of accuracies given gamma and C, there
will be a diagonal of ideal combinations that emerges. The lowest of these is
usually chosen. 



\subsection{Model Selection and Assessment}
\label{sec:selectass}
After a model is trained, the first step is model selection and assessment.
Selection is estimating model performance among a set of trained models using a
single validation set.  After one model is chosen, assessment takes place by
determining the prediction capability on new data via a previously unseen
testing set. Both selection and assessment can be done in a single step using
\textit{k}-fold cross-validation, which is described below.

\subsubsection{Sources of Error} 

In statistical learning, there are two sources of error that need to be
simultaneously minimized: bias and variance. Bias is caused by simplifications
in the model, so the error is caused by missed relationships in the data; an
underfit model is due to high bias.  Variance is caused by including random
noise in the model, so the error is caused by oversensitivity to that noise; an
overfit model is due to high variance. 
\\
\begin{figure}[!htb]
  \includegraphics[width=\linewidth]{./chapters/litrev/BVtradeoff.png}
  \caption{Total prediction error comprised of bias and variance}
  \label{fig:bvtradeoff}
\end{figure}

As shown in Figure \ref{fig:bvtradeoff}, the shape of the total error curve
shows that there is a tradeoff between the bias and variance that can be
minimized. Some bias is desired in order to generalize to future unknown data.
But some variance is also positive for the model because it captures the
relationships in the data that the bias counteracts. 

\subsubsection{Types of Error}

While the sources of the model prediction error are well known, the creation of
a statistically learned model is a hidden process. Although the model emerges
from a black box, there are ways to evaluate the generalization (i.e.,
prediction) capability of it.  This is done by removing a small portion of the
data for use as a testing set.  The rest of the data set is known as the
training set and is used to train a model. After training, the test set is used
to test the model.  

The generalization error is typically referred to as the \textit{testing
error}, as it is measuring the ability of the model to predict future cases
that were not introduced in the training phase (i.e., the testing set entries).
Next, the \textit{training error} is provided by comparing the model
predictions to the training set, as the model would likely be smoother than the
potential noise the training set would include. This is useful to determine the
fitness of the model, the application of which is discussed below in Section
\ref{sec:optvalid}.

Although one could just train and test their model, there is a way to test the
model while still in the training phase. A testing set that would be used
during training to give feedback, a \textit{cross-validation} set, can provide
a faster convergence to a satisfactory model. As shown in Figure
\ref{fig:cverror}, this can be done by splitting the data set into three
groups: a large training set, a small cross-validation set, and a small testing
set. 

\begin{figure}[!htb]
  \includegraphics[width=\linewidth]{./chapters/litrev/cverror.png}
  \caption{Illustration of how a dataset can be split up for model evaluation}
  \label{fig:cverror}
\end{figure}

However, in practice, multiple rounds of cross-validation steps are used,
referred to as \textit{k-fold cross-validation}. This allows a user to use all
data entries as a testing entry once.  As illustrated in Figure
\ref{fig:cverror}, this splits the dataset into \textit{k} subsets. One set is
designated as the testing set, and a model is trained with the rest. Following
the first training phase, another begins, this time with a different subset as
the testing set.  This process is performed \textit{k} times to give \textit{k}
models. The models are `averaged' by taking the mean of the predictions in the
case of regression or voting in the case of classification.  This provides an
additional level of model validation than can be achieved with a single testing
set.



\subsection{Model Optimization and Validation}
\label{sec:optvalid}

\begin{frame}
  \frametitle{Training Set Size: Learning Curves}
  \begin{figure}[h!]
    \centering
    \begin{minipage}{0.35\textwidth}
      \centering
      \includegraphics[width=\linewidth]{./figures/LearningCurve-bias.png}
    \end{minipage}%
    \begin{minipage}{0.35\textwidth}
      \centering
      \includegraphics[width=\linewidth]{./figures/LearningCurve-ideal.png}
    \end{minipage}%
    \begin{minipage}{0.35\textwidth}
      \centering
      \includegraphics[width=\linewidth]{./figures/LearningCurve-variance.png}
    \end{minipage}
    \caption{Learning curves for three training scenarios: high bias, balanced bias and variance, and high variance}
  \end{figure}
\end{frame}

\begin{frame}
  \frametitle{Model Complexity: Validation Curves}
  \begin{figure}[h!]
    \centering
    \includegraphics[height=0.7\textheight]{./figures/ValidationCurve.png}
    \caption{Validation curve showing different fitness of models}
  \end{figure}
\end{frame}

\begin{frame}
  \frametitle{Model Comparison}
  \begin{minipage}{0.5\textwidth}
    \centering
    \begin{table}[h!]
      \centering
      \includegraphics[height=0.7\textheight]{./figures/bayes.png}
      \caption{Bayes}
    \end{table}
  \end{minipage}%
  \begin{minipage}{0.5\textwidth}
    \centering
    \begin{equation}
      Posterior = \frac{Likelihood * Prior}{Marginal \ Likelihood} 
    \end{equation}
  \end{minipage}
\end{frame}



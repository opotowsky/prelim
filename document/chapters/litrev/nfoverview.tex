Nuclear forensics comprises a large part of an investigation into a nuclear
incident, such as interdicted nuclear material or the detonation of a weapon
containing radioactive components.  The forensics portion of the investivation
encompasses both the analysis of nuclear material and/or related paraphrenalia
as well as the interpretation of these results to establish nuclear material
provenance. The former has many technical aspects, relying on a range of
nuclear science and chemistry.  The latter involves intelligence and political
considerations of the material analyses. This review will only consider the
technical portion of the nuclear forensics workflow.

Furthermore, the technical programs researching improvements to the \gls{US}'s
nuclear forensics capabilities are split between the type of material being
investigated. The analysis of irradiated debris from a weapon has different
collection and measurement requirements from recovered \gls{SNF} stolen from a
commercial reactor. This separates the field into post-detonation and pre-detonation 
nuclear forensics. While both are discussed in Sections \ref{sec:postdet} 
and \ref{sec:predet}, respectively, there is more focus on pre-detonation 
topics since this work is based on \gls{SNF}.

\subsection{Post-Detonation Nuclear Forensics}
\label{sec:postdet}


\subsection{Pre-Detonation Nuclear Forensics}
\label{sec:predet}





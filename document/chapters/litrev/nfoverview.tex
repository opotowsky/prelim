Nuclear forensics comprises a large part of an investigation into a nuclear
incident, such as interdicted nuclear material or the detonation of a weapon
containing radioactive components.  The forensics portion of the investigation
encompasses both the analysis of nuclear material and/or related paraphrenalia
as well as the interpretation of these results to establish nuclear material
provenance. The former has many technical aspects, relying on a range of
nuclear science and chemistry.  The latter involves intelligence and political
considerations of the material analyses for attribution. This review will only
consider the technical portion of the nuclear forensics workflow.

Furthermore, the technical programs researching improvements to the \gls{US}'s
nuclear forensics capabilities are split between the type of material being
investigated. The analysis of irradiated debris from a weapon has different
collection and measurement requirements than recovered \gls{SNF} from a
commercial reactor. This separates the field into post-detonation and
pre-detonation nuclear forensics. While both are discussed in Sections
\ref{sec:postdet} and \ref{sec:predet}, respectively, there is more focus on
pre-detonation topics since this work is based on \gls{SNF}.

Additionally, nuclear forensics is a traditional inverse problem, which has
been well documented in mathematics and many scientific disciplines.
Understanding inverse problem theory can help systematically define both the
solution methods and their limitations. An introduction to the topic as well as
its application to nuclear forensics is discussed in \ref{sec:inverse}

\subsection{Types of Nuclear Forensics Investigations}
\label{sec:types}

\subsubsection{Post-Detonation}
\label{sec:postdet}

Post-det

\subsubsection{Pre-Detonation}
\label{sec:predet}

Pre-det

\subsection{Nuclear Forensics as an Inverse Problem}
\label{sec:inverse}

More inverse intro here.

As outlined in \todo{cite inverse theory text}, the study of a typical physical
system encompasses three areas/steps?:
\begin{enumerate}[(i)]
\setlength{\itemsep}{1pt}
\item model parameterization 
\item forward problem predicting measurement values given model parameters 
\item inverse problem predicting model parameters given measurement values 
\end{enumerate}

This shows that it is important to consider the parameters that comprise a
model; this is denoted as the \textit{model space}. This is not every
measurable quantity; domain knowledge is necessary to determine the model
space. In the nuclear forensics context for spent nuclear fuel, this would
consist of, e.g., several isotopic ratios because they are known to have a
relationship with the reactor parameters that created the fuel of interest.
Understanding the physical system also requires an understanding of the forward
problem: predicting how a certain set of model parameters will affect the
resulting measurements. The breadth of these end measurements provides the
\textit{data space}, which are all the conceivable results of a given forward
problem. So for spent nuclear fuel this would be, perhaps, the range of
isotopic ratios typical of a commercial reactor. Lastly, the inverse problem is
statistical in nature: given some solution, there is a probability that the
data measured is caused by some value(s) of a model parameter. Additionally,
including measurement uncertainties broadens the linear model to a probability
density of the parameters. The opposite is also true in the forward case:
including parameter uncertainties broadens the forward problem results to a
probability density of the potential measurement values.

In this way, we can define some probability that the answer is correct, given a
set of measurements and their uncertainty. Inverse problem theory states that
this follows the general form of Bayes' theorem, which is commonly expressed as
follows:
\begin{equation}
P(A|B) = \frac{P(B|A)P(A)}{P(B)}
\end{equation}
where $A$ and $B$ are events, $P(A)$ and $P(B)$ are the probabilities that events
$A$ and $B$ will occur, respectively, $P(B|A)$ is the prior probability that event 
$B$ will occur given a known result for $A$, and $P(A|B)$ is the posterior 
probability that event $A$ will occur given a known result for $B$.

This is can be mapped easily to the inverse physical system problem scenario.
$A$ would represent an occurence of a parameter in the model space, and $B$
would represent the measurement of some value. Thus, $P(A)$ is the probability
of a parameter existing without any knowledge of $B$. This is known as the
likelihood, usually given by some theory about the system. $P(B)$ is the
probability of some measurement existing without any knowledge of $A$. This is
known as the marginal likelihood, which is some homogeneous concept for the
potential measurements that could be made (this only serves to scale to
absolute probabilities and does not affect the relative probabilities). The
prior probability $P(B|A)$ is the chance that a measurement is observed from a
given parameter, representing the forward problem.  Lastly, the posterior
probability is the chance of some parameter existing given some measurement,
representing the inverse problem solution.  It may be more intuitive to
consider the conceptual version of Bayes' theorem below.  
\begin{equation}
Posterior = \frac{Prior * Likelihood}{Marginal \ Likelihood} 
\end{equation} 
A discussion of how these values are obtained takes place in Section
\ref{sec:selection}.

This framework is helpful for an experiment that intends to compare different
methods for calculating the posterior probability of a system given some
measurements.  In the nuclear forensics context of pre-detonated materials,
this would be a a set of probabilities for different parameters of interest,
e.g., reactor type, burnup, cooling time, and enrichment of some interdicted
spent fuel. The prior probabilities are obtained by a large set of forward
problems, e.g., a database of spent fuel recipes and parameters. The
likelihoods are obtained in differing ways. One method is expert-elicited
values. Another is a predicted model from some theory or previously known
relationship, e.g., empirical relations between isotopic ratios and certain
reactor parameters. 

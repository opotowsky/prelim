Although the body of literature on the area of proposed research is not
expansive, there have been a number of relevant studies on the prediction of
forensically relevant categories or quantities of nuclear materials using
statistical methods. 

\subsection{Special Nuclear Materials Studied}

With regards to broader forensics capabilities, materials from different steps
of the nuclear fuel cycle are must be studied.  Even though each material has
its own forensics signatures, the process of applying statistical methods to
the analysis of material provenance is similar for each. 

For example, on the front end of the fuel cycle, an entity may have obtained
\gls{UOC} if they have enrichment capabilities.  One study performed
statistical analyses on \gls{UOC} from 21 sources (throughout seven countries)
using 30 concentration measurements of various elements, isotopes, and
compounds, e.g., sodium, magnesium, thorium, uranium-234, or halide compounds
\cite{robel_2009}.  The goal of classifying the source and the country was
reached 60\% and 85\% of the time, respectively. 
% Note: there was a ton of skew in their data and the correction for that was
% weak sauce

On the back end, an organization might have interest in \gls{SNF} if they have
reprocessing capabilities.  Or, perhaps already separated plutonium from
\gls{SNF} has been intercepted and needs to be traced. Another study addresses
this by performing factor analysis on theoretical separated plutonium from
various sources of \gls{ORIGEN}-simulated \gls{SNF} based on their composition
at the end of irradiation \cite{nicolaou_pu}.  Since in this study all
materials are the same age, five plutonium isotopes (238--242) correctly
predicted a test sample. However, taking different times since irradiation and
reprocessing into account requires more isotopic measurements. 

\subsection{Statistical Methods Employed}

There are many statistical methods studies that focus on classification,
usually the prediction of the reactor type for unknown samples
\cite{robel_2009, nicolaou_pu, jones_snf_2014, nicolaou_2009}.  However, this
work is focused on burnup prediction as well.  Although the results for both
regression and classification are based on a number of features that are
usually isotopic in nature, it is not clear if the regression counterparts of
these algorithms will perform similarly for this task. 



SNF, ~300 features, 3 methods \cite{dayman_feasibility_2013} (burnup)
nnl1, nnl1, rr

Nico 2006 predicted enrichment and burnup using factor analysis + nico 2015 - regression only
factor analysis
%SNF, 9 features, 2 methods \cite{pu_discrimination} (fuel enrichment)
%PCA + PLSDA


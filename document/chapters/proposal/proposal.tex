{Research Proposal}
\label{ch:proposal}

This document previously demonstrated the performance of \gls{ML} on a
set of nuclear material isotopics to calculate a reactor parameter of interest:
burnup.  Additionally, the procedures for understanding the learned model's
behavior and thus the quality of the results was discussed.  Next, a set of
experiments are presented here. 

Before describing the experiments, some preparatory topics and issues are
addressed in Section \ref{sec:prep}.  Next, the proposed research experimental
design is presented. Sections \ref{sec:exp1}, \ref{sec:exp2}, and
\ref{sec:exp3} describe evaluating the viability of statistical learning using
direct isotopic information, information obtained from gamma spectra, and other
fuel cycle flows, respectively.  Discussed separately is the planned
implementation of comparing methods using the Bayesian framework in Section
\ref{sec:modelcompare}. Finally, the timeline of this project is covered in
Section \ref{sec:timeline}.

\section{Experiment Preparations}
\label{sec:prep}

\subsection*{Expanding Training Set}

As identified in Section \ref{sec:valid}, the testing set used for the
demonstration was not suitable for further study without being expanded. Also
mentioned was that many algorithms are developed on an assumption that the
training set will be \acrfull{i.i.d.}.  This is important so that the model
does not overvalue or overfit a certain area in the training space.  The next
step is to provide a larger, more diverse training set to the algorithms so
they can better predict labels when faced with new instances. This diversity
will be suggested from the \gls{SFCOMPO} database \cite{sfcompo}, as it
includes many common domestic and international reactors.

The SFCOMPO-2.0 relational database \cite{sfcompo} has approximately 750
chemical assay measurements of \gls{SNF} from 44 reactors. While this is not
sufficient as a training set, it provides a better framework for a training
set. It will be simulated with \gls{ORIGEN} as inspired by the breadth of
values in the \gls{SFCOMPO} database. After \gls{CV}, diagnostics, and
optimization, the \gls{ML} models can be tested against the entries in this
database to provide a clear estimate of of the model performance. 

\subsection*{Finalizing Set of Algorithms}

The three algorithms in the demonstration (\textit{k}-nearest neighbor, ridge,
and \gls{SVR}) are not necessarily the set that will be evaluated for the
experiments. After these are used to train new models on a larger training set
for comparison, other algorithms can be speedily assessed as well. The
scikit-learn toolkit allows for easy transitions because the algorithms are
already implemented; applying a new \gls{ML} algorithm involves merely changing
the method.  Since support vector-, distance-, and linear-based models are
already represented, other obvious choices include Bayesian methods, decision
trees, neural networks, or ensemble methods \cite{elements_stats}. 

\subsection*{Computational Framework and Resources}

Thus far, all simulations and training have not required more processing than
available on a personal computer. However, some algorithms do require larger
amounts of computational time (e.g., neural networks).  If necessary, the
training stage can be done using the \gls{CHTC}, which is available to
University of Wisconsin researchers. 

\section[Experiment 1: Direct Isotopics]{Experiment 1\\ 
\large{\textit{Viability of Statistical Learning on Direct Isotopics}}}
\label{sec:exp1}

The first experiment will be a purposefully constructed version of the
demonstration: evaluating the model performance with known isotopics.  This
sheds light on how this methodology will perform on the simplest scenario,
providing an upper estimate of performance.  \textit{The main purpose of this
experiment is to probe the usefulness of statistical methods for
determining reactor parameters, ultimately choosing the best performing
methods.}

\begin{figure}[!htb]
    \centering
    \includegraphics[width=\linewidth]{./chapters/proposal/proposal.png}
    \caption{Motivation for Data Sets for Experiments 1 and 2}
    \label{fig:proposal}
\end{figure}

Figure \ref{fig:proposal} shows what type of input can be used for training and
prediction.  The two horizontal boxes are the physical and computational forms
of what these experiments are simulating, respectively. The lab-measured mass
spectra correspond to the perfect information being referred to here. In the
computational context, these measurements instead come from simulations. Mass
spectra results are thus approximated as the direct isotopics given from the
simulations, since they provide highly reliable and accurate information.

The variables for this experiment will include the following:
\begin{enumerate}
  \itemsep-0.75em
  \item the complexity of the \gls{ML} algorithm used, 
  \item feature reduction, and 
  \item different subsets of the decision space.
\end{enumerate}

As previously mentioned, there are complexity differences among the algorithms,
which will be taken into account.  The distance-based nearest neighbor
algorithm is by far the simplest, as it does not do any learning. By contrast,
\gls{SVR} with a radial basis function kernel to enable learning in higher
orders is much more elaborate.  The feature reduction will be different subsets
of isotopes, e.g., top \textit{n} nuclides by concentration, actinides only,
fission products only, etc.  Reducing the decision space can be done by, e.g.,
simplifying the regression task by fixing the reactor type. It is expected that
a more complex algorithm (e.g., \gls{SVR}) will be needed, and that
preprocessing and/or manual feature reduction will assist in creating higher
quality models.  Simplifying the decision space should always improve
prediction, but it is not obvious how much it will impact burnup prediction
specifically.

\subsection*{Risks}

It is possible that statistical models trained on direct isotopic information
do not perform well enough.  Other than attempting different types of
algorithms, the next tuning would occur by preprocessing the data further,
statistically performing feature reduction via \gls{PCA}. If this is not
sufficient, \gls{SNF} may have too many or too few correlated features to
provide reliable models across the space of current reactor technologies. Since
separated plutonium and \gls{UOC} have been also studied using these
techniques, these materials could provide an alternate route to obtain useable
learned models. Additionally, this methodology would also work if applied to
post-detonation materials. There is work on creating standard materials to
represent the ``urban canyon''. The noise from irradiated concrete, building
materials, etc. is expected to pose a great challenge to forensics teams. This
is another evolving subject that may benefit from statistical correlations
\cite{refmaterial}.

\section[Experiment 2: Gamma Spectra]{Experiment 2\\ 
\large{\textit{Viability of Statistical Learning on Gamma Spectra}}}
\label{sec:exp2}

The second experiment will be the previously discussed extension of the
demonstration by applying \glspl{DRF} to the gamma energies: evaluating the
model performance with reduced isotopic information.  This demonstrates the
usefulness of this methodology in a real-world scenario where exact isotopics
are not always known.   \textit{The main objective of this experiment is to
measure the reduction in statistical model parameter prediction reliability as
the quality of the training information is reduced.}

The two bottom portions of the boxes in Figure \ref{fig:proposal} represent a
more realistic measurement scheme, involving a model trained from gamma
spectrometers rather than the lengthy process of performing mass spectrometry
on the samples.  In the physical context, the measurements for training would
be done using a semiconductor gamma detector, but the testing or prediction
step may be done outside of the lab on a different detector.  This will be
captured by applying different \glspl{DRF} to the radionuclide
inventories from the simulations.

The variables for this experiment will include the following:
\begin{enumerate}
  \itemsep-0.75em
  \item the complexity of the \gls{ML} algorithm used, 
  \item feature reduction (implicit), and 
  \item quality of training and/or testing data set.
\end{enumerate}
As in Section \ref{sec:exp1}, performance will be explained with respect to
algorithm complexity.  The feature reduction is implicit here, since gamma
detection only includes radionuclides within the \gls{SNF} isotopics. The
indirect isotopic training data are likely going to reduce the prediction
capability of the models, but it is not yet clear if a response function
simulating a hand-held NaI gamma detector can provide any useful predictions.
And while it is still expected that the complex algorithms will perform better,
it is not yet obvious if different algorithms than the ones used in Experiment
1 will be needed. 

\subsection*{Risks}

Following the risks in Section \ref{sec:exp1},it is possible that statistical models trained on indirect isotopic information
do not perform well enough. Different algorithms may perform
better than others due to the underlying optimization processes. Further
feature reduction could also prove useful, focusing on particular energy
regions or particular peaks throughout the spectrum. The nature of the
training information could be altered slightly by using an isotope
identification algorithm; this may improve the performance, as the the typical field detectors are
developed to automatically report isotopics from gamma spectra. If this still
is not sufficient, it may be that direct isotopic information (i.e., that
obtained from mass spectrometry) is required for reliable statistical models of
\gls{SNF}.  Lastly, preprocessing could also be investigated here.

\section[Experiment 3: Other Fuel Cycle Flows]{Experiment 3\\ 
\large{\textit{Viability of Statistical Learning on Reprocessed Fuel}}}
\label{sec:exp3}

There is already nuclear fuel reprocessing in France for \gls{LWR}s, and
possibly in the forseeable future in China with fast breeder reactors.  Given
the existence of reprocessing in fuel cycles, it is important to develop
nuclear forensics capabilities on processed \gls{SNF}. This is presents an
additional prediction challenge because many of the isotopic, chemical, and
elemental signatures are stripped away in the processing. However, there is a
possibility that plutonium isotopes can provide enough information for
characterization, even with reprocessing \cite{pu_discrimination}. Although
Ref. \cite{pu_discrimination} predicts reactor type and enrichment, it is
enough motivation to pursue burnup as well. 

This experiment is intended to repeat those in Sections \ref{sec:exp1} and
\ref{sec:exp2} with an added layer of complexity from the reprocessing by
including mixed-oxide fuel.  It will be carried out using direct isotopics as
in Section \ref{sec:exp1}, with a goal to use indirect information as in
Section \ref{sec:exp2} pending performance. The training data set will also be
simulated using \gls{ORIGEN}.  \textit{The main purpose of this experiment is
to probe the difficulties of reactor parameter prediction when some of the
inputs are from reprocessed nuclear fuel.}

The variables for this experiment will include the following:
\begin{enumerate}
  \itemsep-0.75em
  \item the complexity of the \gls{ML} algorithm used,
  \item quality of training data set, and 
  \item type of preprocessing for feature reduction.
\end{enumerate}

Because of the increased complexity of the training data, it is expected that a
more involved algorithm will be necessary.  The quality of the training data
set will be varied as well as the type of preprocessing tool.  It is not yet
obvious if the feature reduction is absolutely necessary, but it is expected to
improve the accuracy of burnup prediction.  Feature reduction can
be carried out via a range of discriminant, component, or factor analyses.
Multiple studies have demonstrated success with some of these dimensionality
reduction techniques \cite{nicolaou_2006, nicolaou_2009, nicolaou_2014,
robel_2009, pu_discrimination, jones_viz_2014, jones_snf_2014}.  Because
reprocessing mixes streams of material, \gls{PCA} is not expected to perform as
well as \gls{ICA}. This was discussed in Section \ref{sec:dimreduc}.

\subsection*{Risks}

It is possible that this framework is not the best approach to understand fuel
cycles with reprocessing. The first action upon poor performance results is to
try more/different preprocessing tools; this may have to include using domain
knowledge to manually reduce the dimensions. As one of the goals within the
nuclear forensics community is to identify new signatures and/or correlated
measurements, this is not desirable but it could still offer interesting
results. Also, some creativity could be applied to find a method that works for
this type of dataset, such as the iterative partial least squares discriminant
analysis used for determining \gls{UOC} origin country and source prediction in
Refs.  \cite{robel_2009, pu_discrimination}.

\section{Method Comparison}
\label{sec:modelcompare}

It is essential to be able to compare the models proposed here against each
other, but also against other analagous non-statistical methods.  For example,
Experiment 2 (Section \ref{sec:exp2}) aims to predict reactor history values
using simulated gamma spectra, and \gls{INDEPTH} uses gamma energies/spectra as
inputs.  These can be directly compared using the inverse problem structure
introduced in Section \ref{sec:inverse}. An application of Equation
\ref{eq:bayes_words} is discussed further in Section \ref{sec:invcompare}.

However, the Section \ref{sec:invcompare} discussion excluded some detail for
clarity. While initially uncertainties will be ignored for an approximate
analysis, they cannot be ignored for long in the nuclear forensics context.
These are quite important for obtaining some measure of confidence in the
solution.  Including the uncertainty broadens each predicted parameter to a
range of probabilities.  These are then analyzed to produce ranges of
confidence in each prediction \cite{bayes_compare}.  This means that the
posterior probabilities discussed will not just be numbers, but probability
distributions.  The methods for calculating the necessary probability
distributions are outlined below.

Here, we change the meaning of the variables to represent probability
distributions, shown in Equation \ref{eq:distrib}.  $C$ is a constant given by
the marginal likelihood, which can be ignored when calculating relative
probabilities, and $\boldsymbol{d}$ and $\boldsymbol{m}$ represent the training
data set and model parameters, respectively. Thus,
$P(\boldsymbol{d}|\boldsymbol{m})$ is the likelihood distribution function,
$P(\boldsymbol{m})$ is the prior probability distribution, and
$P(\boldsymbol{m}|\boldsymbol{d})$ is the posterior probability distribution.
\begin{equation}
\label{eq:distrib}
  P(\boldsymbol{m}|\boldsymbol{d}) = C\ *\
  P(\boldsymbol{d}|\boldsymbol{m})\ *\ P(\boldsymbol{m})
\end{equation}

Mathematically speaking, the distributions are obtained by integrating over the
relevant probability densities.  For example, the prior probability
distribution can be calculated from Equation \ref{eq:rho}, where
$\boldsymbol{m}$ is the range of predicted model parameters, i.e. burnup
values, and $\boldsymbol{d}$ is a set of nuclide vectors. Also, here,
$\rho(\boldsymbol{x}) = \prod_{i} \rho(x_i)$. 
\begin{equation}
\label{eq:rho}
  P(\boldsymbol{m}) = \int_{\boldsymbol{m}} \rho(\boldsymbol{d}) d\boldsymbol{d}
\end{equation}
Similarly, the likelihood distribution function is obtained as in Equation 
\ref{eq:rho-l}.
\begin{equation}
\label{eq:rho-l}
  P(\boldsymbol{d}|\boldsymbol{m}) = \int_{\boldsymbol{d}, \boldsymbol{m}} \rho(\boldsymbol{d}|\boldsymbol{m}) d\boldsymbol{m}
\end{equation}
In practice, however, these density functions are not calculated directly.
Below, the methods chosen to estimate the functions in this work are addressed.

\subsection{Estimating Density Functions}

Estimating a probability density function $\rho(x)$ is not a straightforward
task.  One must have a sense for the shape of this function to predict a smooth
curve that will represent the probability density over a range of a parameter.
One estimation method is binning the parameters in the training set over a
given width and tallying up each bin to form a histogram; using some kernel,
Gaussian for example, provides a function that should approximately follow the
shape of the histogram. The shape of the histogram and function may strongly
depend on the bin size, but this can be handled as a separate optimization step
within the sci-kit learn package \cite{scikit}.

The prior probability distributions are given by the model space, e.g., reactor
parameters as predicted from the \gls{ML} models.  The above-mentioned
method should be utilized to obtain a prior density function from the range of
burnups.  This will not be necessary because the current design of the training
set will provide a nearly flat density function (i.e., uniform distribution).
However, this is not the case in a real-world scenario with mostly commercial
reactor fuel comprising \gls{SNM} inventories around the world.  This topic
will only be investigated if the training data set is changed.
\cite{bayes_compare} Note: This implies the posterior is now only dependent on
the likelihood.

The likelihood function can be obtained by summing the probability densities of
each nuclide vector ($d$) given some predicted model parameter ($m$, e.g., a
burnup prediction) using some algorithm (e.g., nearest neighbor regression).
In essence, this is a large set of forward problems; it is information
including the database of \gls{SNF} recipes, and the known parameter inputs
compared to the model's parameter predictions.  The goal of this step is to
find the \textit{maximum} likelihood, usually carried out in a step called
\gls{MLE}. In this framework, the training phase provides the maximum
likelihood distribution through the use of \gls{CV}, since the results
are reported as a mean error with a standard deviation (which can be
converted to accuracy for likelihood) \cite{scikit}. 

Unfortunately, \gls{MLE} is not this simple for other methods that do not
employ \gls{CV} \cite{gentle_bayes, bayes_compare}. This will indeed
have to be broached in order to compare the machine-learned models against
other forensics methods but has not yet been demonstrated (in progress).

\subsection{Posterior Odds}

Finally, a non-normalized posterior probability distribution, $P(m_i|d)$, can
be calculated directly from the likelihood and prior distributions. The same
can be done for a model obtained from a different algorithm, $P(m_j|d)$. The
relative posterior probability distribution, aka \textit{posterior odds}
\cite{bayes_compare}, can then be obtained as in Equation \ref{eq:postodds}.
Here, $B_{ij} = \frac{\rho(d|m_i)}{\rho(d|m_j)}$ and is referred to as the
\textit{Bayes factor}.
\begin{equation}
\label{eq:postodds}
  \frac{P(m_i|d)}{P(m_j|d)} = B_{ij} \frac{P(m_i)}{P(m_j)}
\end{equation}

Given somewhat uniform priors, the Bayes factor is the key component that
determines relative model performance. Table \ref{tbl:strength}\footnote{This
table is reproduced from Ref. \cite{bayes_compare}.} shows an approximation of
posterior probabilities calculated from $\lvert lnB_{ij} \rvert$. Taking the
logarithm of the Bayes factor is done for convenience, since taking the product
of many likelihoods tends towards zero.

\begin{table}[!hbt]
  \centering
  \includegraphics[width=0.7\linewidth]{./chapters/proposal/evidence-strength.png}
  \caption{Model Comparison using Likelihood Strength}
  \label{tbl:strength}
\end{table}

In summary, given a mean-squared error and its standard deviation from using
\gls{CV} with any \gls{ML} algorithm, the \gls{MLE} can be formed. To
compare two models, a ratio of $\text{MLE}_i$ to $\text{MLE}_j$, called the
posterior odds, provides the probability of model $i$ being correct.

\section{Timeline}
\label{sec:timeline}

Figure \ref{fig:time} is an approximate timeline over which this work will
occur.  Multiple tasks will be complete before my internship is finished, and
much of the computational framework is also complete. The experiments are
expected to take a few weeks each, accounting for time spent on diagnostics and
readying the results for the next step. Formal validation and model comparison
will then be carried out. Because of reporting requirements, some preparatory
writing will also be done during the internship. Thus, final analyses and
writing are expected to take three months. 

\begin{sidewaysfigure}[!htb]
  \makebox[\textwidth][c]{\includegraphics[width=1.2\textwidth]{./chapters/proposal/timeline.png}}  
  \caption{Timeline for Project}
  \label{fig:time}
\end{sidewaysfigure}


Nuclear forensics is a nuclear security capability that is broadly defined as
material attribution in the event of a nuclear incident.  Improvement and
research is needed for both technical and non-technical components of this
process.  One such technical area is the provenance of non-detonated \gls{SNM};
studied here is \gls{SNF}, which is applicable in a scenario involving the
unlawful use of commercial byproducts from nuclear power reactors.  The
experimental process involves measuring known forensics signatures to ascertain
the reactor parameters that produced the material. Knowing these assists in
locating the source of the material. This work is proposing the use of
statistical methods to determine these quantities instead of empirical
relationships. 

The purpose of this work is to probe to what extent this method is feasible.
Thus, three experiments have been designed, using nuclide vectors as
observations and simulation inputs as the prediction goals.  First, machine
learning algorithms will be employed with full-knowledge training data, i.e.,
nuclide vectors directly from simulations.  Second, this workflow will be
performed on reduced-knowledge training data, analgous to a detector that can
only measure certain radionuclides. Third, these two experiments will then
include recycled nuclear fuels to evaluate the methodology when many chemical
and elemental signatures are lost in processing. The results will be evaluated
using both the prediction accuracy and confidence.

Nuclear forensics is a nuclear security capability that is broadly defined as
material attribution in the event of a nuclear incident.  Improvement and
research is needed for both technical and non-technical components of this
process.  One such technical area is the provenance of non-detonated \gls{SNM};
studied here is \gls{SNF}, which is applicable in a scenario involving the
unlawful use of commercial byproducts from nuclear power reactors.  The
experimental process involves measuring known forensics signatures to ascertain
the reactor parameters that produced the material. Knowing these assists in
locating the source of the material. This work is proposing the use of
statistical methods to determine these quantities instead of evaluating
empirical relationships. 

The purpose of this work is to probe to what extent this computational workflow
is feasible. Thus, three experiments have been designed, using nuclide vectors
as observations and simulation inputs as the prediction goals.  First, machine
learning algorithms will be employed with training data that is considered to
be perfect knowledge, i.e., nuclide vectors directly from simulations.  Second,
machine learning will be performed on less perfect training data, analgous to a
detector that can only measure certain radionuclides. Third, the previous two
experiments will include recycled nuclear fuels to evaluate if the workflow is
useful when many chemical and elemental signatures are lost in processing. The
results of the experiments will be evaluated by the prediction accuracy as well
as the prediction confidence.


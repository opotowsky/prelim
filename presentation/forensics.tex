
\begin{frame}
  \frametitle{Nuclear Forensics Investigations}
  \begin{minipage}[t]{0.5\textwidth}
    \textbf{Post-detonation}
    \begin{itemize}
      \item Collection: debris, swipe samples
      \item Characterization: rapid analysis of isotope ratios
      \item Goals
      \begin{itemize}
        \item Inverse problem: reconstruct weapon design/yield
        \item Safety: informing disaster response
      \end{itemize}
      \item Data evaluation
    \end{itemize}
  \end{minipage}%
  \pause
  \begin{minipage}[t]{0.5\textwidth}
    \textbf{Pre-detonation}
    \begin{itemize}
      \item Collection: depends on intercepted material
      \item \boxalert{Characterization:} non-destructive and destructive
      \item Goals:
      \begin{itemize}
        \item \boxalert{Inverse problem:} material chain of custody
        \item Safety: material handling and security
      \end{itemize}
      \item \boxalert{Data evaluation}
    \end{itemize}
  \end{minipage}
\end{frame}


\begin{frame}
  \frametitle{Nuclear Forensics as an Inverse Problem}
  Necessary to determine the quality of prediction

  Use Bayes' Framework:
  $$ P(A|B) = \frac{P(B|A)P(A)}{P(B)} $$
  $$ P(M|D) = \frac{P(D|M)P(M)}{P(D)} $$
\end{frame}

